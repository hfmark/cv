%%%%%%%%%%%%%%%%%%%%%%%%%%%%%%%%%%%%%%%%%%%%%%%%%%%%%%%%%%%%%%%%%%%%%%%%%%%%%%%
% A clean template for an academic CV. This is a short summary version.
%
% Uses tabularx to create two column entries (date and job/edu/citation).
% Defines commands to make adding entries simpler.
%
%%%%%%%%%%%%%%%%%%%%%%%%%%%%%%%%%%%%%%%%%%%%%%%%%%%%%%%%%%%%%%%%%%%%%%%%%%%%%%%

\documentclass[10pt, letterpaper]{article}

% Full Unicode support for non-ASCII characters
\usepackage[utf8]{inputenc}

% Useful aliases
\newcommand{\WHOI}{Woods Hole Oceanographic Institution}
\newcommand{\WUSTL}{Washington University in St Louis}
\newcommand{\UChi}{University of Chicago}
\newcommand{\JP}{MIT-WHOI Joint Program}
\newcommand{\MIT}{Massachusetts Institute of Technology}
\newcommand{\mathrev}{WHOI Summer Math Review}
\newcommand{\AGU}{AGU Fall Meeting}

% Identifying information
\newcommand{\Title}{Curriculum Vit\ae}
\newcommand{\FirstName}{Hannah}
\newcommand{\LastName}{Mark}
\newcommand{\MiddleName}{F.}
\newcommand{\MyName}{Dr. \FirstName\ \MiddleName\ \LastName}
\newcommand{\Initials}{HF}  % for citations in particular
\newcommand{\Me}{\textbf{\LastName, \Initials}}  % For citations
\newcommand{\Email}{hmark@whoi.edu}
\newcommand{\PersonalWebsite}{hfmark.github.io}
%\newcommand{\LabWebsite}{www.compgeolab.org}
\newcommand{\ORCID}{0000-0002-1722-3759}
\newcommand{\Address}{
  266 Woods Hole Road MS\#24, Woods Hole, MA, 02543
}

% Names for citing coauthors
\newcommand{\Josh}{Russell, JB}
\newcommand{\Jim}{Gaherty, JB}
\newcommand{\Greg}{Hirth, G}
\newcommand{\LHans}{Hansen, LN}
\newcommand{\danl}{Lizarralde, D}
\newcommand{\jac}{Collins, JA}
\newcommand{\behn}{Behn, MD}
\newcommand{\RobE}{Evans, RL}
\newcommand{\Walid}{Ben Mansour, W}
\newcommand{\Doug}{Wiens, DA}
\newcommand{\Russo}{Russo, RM}
\newcommand{\ARich}{Richter, A}
\newcommand{\EMard}{Marderwald, E}
\newcommand{\Rodrigo}{Adaros, R}
\newcommand{\SBarr}{Barrientos, S}
\newcommand{\Ivins}{Ivins, ER}
\newcommand{\Bix}{Magnani, MB}
\newcommand{\Billy}{Shinevar, WJ}
\newcommand{\Fiona}{Clerc, F}
\newcommand{\Emman}{Codillo, EA}
\newcommand{\Jianhua}{Gong, J}
\newcommand{\jao}{Olive, JA}
\newcommand{\SBrow}{Brown, SM}
\newcommand{\PSmal}{Smalls, PT}
\newcommand{\Yang}{Liao, Y}
\newcommand{\Vero}{Le Roux, V}
\newcommand{\Nate}{Miller, NC}
\newcommand{\Yajing}{Liu, Y}
\newcommand{\Donna}{Shillington, D}
\newcommand{\Ari}{Cortes Rivas, V}
\newcommand{\JEst}{Estep, JD}
\newcommand{\Patrickelder}{Shore, P}
\newcommand{\Maeva}{Pourpoint, M}
\newcommand{\PLin}{Lin, PP}
\newcommand{\ESar}{Sarafian, EK}
\newcommand{\ZMa}{Ma, Z}
\newcommand{\CDal}{Dalton, CA}
\newcommand{\Dion}{Heinz, DL}
\newcommand{\Masako}{Tominaga, M}
\newcommand{\WSage}{Sager, W}
\newcommand{\CRowe}{Rowe, C}
\newcommand{\MAgi}{Agius, M}
\newcommand{\JConv}{Convers, J}
\newcommand{\GFun}{Funning, G}
\newcommand{\CGal}{Galasso, C}
\newcommand{\SHic}{Hicks, S}
\newcommand{\THuy}{Huynh, T}
\newcommand{\JLan}{Lange, J}
\newcommand{\TLec}{Lecocq, T}
\newcommand{\ROku}{Okuwaki, R}
\newcommand{\Thea}{Ragon, T}
\newcommand{\CRyc}{Rychert, C}
\newcommand{\STep}{Teplitzky, S}
\newcommand{\MvdE}{van den Ende, M}
\newcommand{\EKar}{Karasözen, E}
\newcommand{\RDCF}{Corona-Fernandez, RD}



% Template configuration
%%%%%%%%%%%%%%%%%%%%%%%%%%%%%%%%%%%%%%%%%%%%%%%%%%%%%%%%%%%%%%%%%%%%%%%%%%%%%%%

% Disable hyphenation
\usepackage[none]{hyphenat}

% Control the font size
\usepackage{anyfontsize}

% Icon fonts (requires using xelatex or luatex)
\usepackage[fixed]{fontawesome5}
\usepackage{academicons}

% Template variables for styling
\newcommand{\TablePad}{\vspace{-0.4cm}}
\newcommand{\SoftwareTitle}[1]{{\bfseries #1}}
\newcommand{\TableTitle}[1]{{\fontsize{12pt}{0}\selectfont \itshape #1}}

% For fancy and multipage tables
\usepackage{tabularx}
\usepackage{ltablex}

% Define a new environment to place all CV entries in a 2-column table.
% Left column are the dates, right column the entries.
\usepackage{environ}
\NewEnviron{EntriesTable}{
\TablePad
\begin{tabularx}{\textwidth}{@{}p{0.12\textwidth}@{\hspace{0.02\textwidth}}p{0.86\textwidth}@{}}
  \BODY
\end{tabularx}
}

% Macros to add links and mark publications
\newcommand{\DOI}[1]{doi:\href{https://doi.org/#1}{#1}}
\newcommand{\dataDOI}[2]{\href{https://doi.org/#2}{#1}}
\newcommand{\DOILink}[1]{\href{https://doi.org/#1}{doi.org/#1}}
\newcommand{\Website}[1]{\href{https://#1}{#1}}
\newcommand{\Preprint}[1]{\newline Preprint: \faFilePdf\ \DOILink{#1}}
\newcommand{\Youtube}[1]{\newline Recording: \faYoutube\, \href{https://www.youtube.com/watch?v=#1}{youtube.com/watch?v=#1}}
\newcommand{\GitHub}[1]{\faGithub\ \href{https://github.com/#1}{github.com/#1}}
\newcommand{\Role}[1]{\newline Role: \faUsers\ #1}
\newcommand{\Slides}[1]{\newline Slides: \faTv\ \href{https://#1}{#1}}
\newcommand{\SlidesDOI}[1]{\newline Slides: \faTv\ \DOILink{#1}}
\newcommand{\PosterDOI}[1]{\newline Poster: \faImage\ \DOILink{#1}}
\newcommand{\OA}{\aiOpenAccess}

% Macros to set the year and duration on the left column
\newcommand{\Duration}[2]{\fontsize{9pt}{0}\selectfont #1 -- #2}
\newcommand{\Year}[1]{\fontsize{9pt}{0}\selectfont #1}
\newcommand{\Ongoing}{on}  % or use 'on'?
\newcommand{\Future}{future}
\newcommand{\Appointment}[4]{\textbf{#1} \newline #2 \newline #3 \newline #4}

% Define command to insert month name and year as date
\usepackage{datetime}
\newdateformat{monthyear}{\monthname[\THEMONTH], \THEYEAR}

% Set the page margins
\usepackage[letterpaper,margin=1in,includehead,headsep=5mm]{geometry}

% To get the total page numbers (\pageref{LastPage})
\usepackage{lastpage}

% No indentation
\setlength\parindent{0cm}

% Increase the line spacing
\renewcommand{\baselinestretch}{1.2}
% and the spacing between rows in tables
\renewcommand{\arraystretch}{1.2}

% Remove space between items in itemize and enumerate
\usepackage{enumitem}
\setlist{nosep}

% Use custom colors
\usepackage[usenames,dvipsnames]{xcolor}

% Set fonts (requires compilation with xelatex)
\usepackage{fontspec}
\setmainfont[%
  Path = fonts/notoserif/,
  UprightFont = NotoSerif-Regular,
  BoldFont = NotoSerif-Bold,
  ItalicFont = NotoSerif-Italic,
  Extension = .ttf
]{NotoSerif}

% Set the spacing for sections
\usepackage{titlesec}
\titleformat{\section}
  {\normalfont\large\mdseries} % format
  {} % label
  {0pt} % separation (left separation for hang)
  {} % text before title
  [\titlerule] % text after title
\titleformat{\subsection}
  {\normalfont\mdseries} % format
  {} % label
  {0pt} % separation (left separation for hang)
  {} % text before title

% Disable number of sections. Use this instead of "section*" so that the sections still
% appear as PDF bookmarks. Otherwise, would have to add the table of contents entries
% manually.
\makeatletter
\renewcommand{\@seccntformat}[1]{}
\makeatother

% Set fancy headers
\usepackage{fancyhdr}
\pagestyle{fancy}
\fancyhf{}
%\lhead{\fontsize{9pt}{10pt}\selectfont
%  \monthyear\today
%}
%\chead{
%  \fontsize{9pt}{10pt}\selectfont
%  \MyName
%  \hspace{0.2cm} -- \hspace{0.2cm}
%  \Title
%}
%\rhead{\fontsize{9pt}{10pt}\selectfont \thepage{} of \pageref*{LastPage}}
\renewcommand{\headrulewidth}{0pt}

% Metadata for the PDF output and control of hyperlinks
\usepackage[colorlinks=true]{hyperref}
\hypersetup{
  pdftitle={\MyName\ - \Title},
  pdfauthor={\MyName},
  linkcolor=blue,
  citecolor=blue,
  filecolor=black,
  urlcolor=MidnightBlue
}
%%%%%%%%%%%%%%%%%%%%%%%%%%%%%%%%%%%%%%%%%%%%%%%%%%%%%%%%%%%%%%%%%%%%%%%%%%%%%%%


\begin{document}

%%%%%%%%%%%%%%%%%%%%%%%%%%%%%%%%%%%%%%%%%%%%%%%%%%%%%%%%%%%%%%%%%%%%%%%%%%%%%%%

% No header for the first page
\thispagestyle{empty}

% Remove extra header space
\vspace*{-2.75\baselineskip}

\begin{minipage}[t]{0.6\textwidth}
  {\fontsize{16pt}{0}\selectfont\MyName}
\end{minipage}
\begin{minipage}[t]{0.4\textwidth}
  \begin{flushright}
    Last updated: \monthyear\today
  \end{flushright}
\end{minipage}
\\[-0.1cm]
\rule{\textwidth}{1pt}
\vspace{-0.75cm}
\begin{center}
  Email: \href{mailto:\Email}{\Email} |
  ORCID: \href{https://orcid.org/\ORCID}{\ORCID} |
  Website: \Website{\PersonalWebsite}
  \\
  \Address
\end{center}
\vspace{-0.5cm}


%%%%%%%%%%%%%%%%%%%%%%%%%%%%%%%%%%%%%%%%%%%%%%%%%%%%%%%%%%%%%%%%%%%%%%%%%%%%%%%
\section{Education}

\begin{EntriesTable}
  \Year{2019}  &
  \textbf{PhD in Marine Geophysics}, \JP
  \\
  \Year{2014}  &
  \textbf{BA Physics, BS Geophysical Sciences}, \UChi
\end{EntriesTable}


%%%%%%%%%%%%%%%%%%%%%%%%%%%%%%%%%%%%%%%%%%%%%%%%%%%%%%%%%%%%%%%%%%%%%%%%%%%%%%%
\section{Professional Appointments}

\begin{EntriesTable}
  \Duration{2021}{\Ongoing}  &
  \textbf{Postdoctoral Investigator}, \WHOI
  \\
  \Duration{2019}{2021}  &
  \textbf{Fossett Postdoctoral Fellow}, \WUSTL
%  \\
%  \Year{2013}  &
%  \textbf{Summer Student Fellow}, \WHOI
%  \\
%  \Duration{2011}{2014}  &
%  \textbf{Undergraduate Research Assistant}, \UChi
\end{EntriesTable}


%%%%%%%%%%%%%%%%%%%%%%%%%%%%%%%%%%%%%%%%%%%%%%%%%%%%%%%%%%%%%%%%%%%%%%%%%%%%%%%
\section{Publications}

\begin{EntriesTable}
\Year{2023}  &
  \Me, \danl, \Doug.
  Constraints on bend-faulting and mantle hydration at the Marianas Trench from seismic anisotropy.
  \emph{Geophysical Research Letters.}
  \DOI{10.1029/2023GL103331} \newline
  Open data: \dataDOI{scripts for anisotropy calculations}{10.5281/ZENODO.7105231}
  \\
\Year{2022}  &
  \Josh, \Jim, \Me, \Greg, \LHans, \danl, \jac, \RobE. 
  Seismological evidence for girdled olivine lattice-preferred orientation in oceanic lithosphere and implcations for mantle deformation processes during seafloor spreading. 
  \emph{Geochemistry, Geophysics, Geosystems.}
  \DOI{10.1029/2022GC010542}
  \\
\Year{2022}  &
  \Walid, \Doug, \Me, \Russo, \ARich, \EMard, \SBarr. 
  Mantle flow pattern associated with the Patagonian slab window determined from azimuthal anisotropy. 
  \emph{Geophysical Research Letters.}
  \DOI{10.1029/2022GL099871}
  \\
\Year{2022}  &
  \Me, \Doug, \Ivins, \ARich, \Walid, \Bix, \EMard, \Rodrigo, \SBarr.
  Lithospheric erosion in the Patagonian slab window, and implications for glacial isostasy.
  \emph{Geophysical Research Letters.}
  \DOI{10.1029/2021GL096863} \newline
  Open data: \dataDOI{velocity and viscosity models}{10.5281/zenodo.5794167}, and some \dataDOI{cross-correlations}{10.5281/zenodo.5508198}
  \\
\Year{2021}  &
  \Me, \jac, \danl, \Greg, \Jim, \RobE, \behn.
  Constraints on the depth, thickness, and strength of the G discontinuity in the central Pacific from S receiver functions.
  \emph{Journal of Geophysical Research: Solid Earth.}
  \DOI{10.1029/2019JB019256}
  \\
\Year{2019}  &
  \Billy, \Me, \Fiona, \Emman, \Jianhua, \jao, \SBrow, \PSmal, \Yang, \Vero, \behn.
  Causes of oceanic crustal thickness oscillations along a 74-Myr Mid-Atlantic Ridge flow line.
  \emph{Geochemistry, Geophysics, Geosystems.}
  \DOI{10.1029/2019GC008711}
  \\
\Year{2019}  &
  \Me, \danl, \jac, \Nate, \Greg, \Jim, \RobE.
  Azimuthal seismic anisotropy of 70 Ma Pacific-plate upper mantle. 
  \emph{Journal of Geophysical Research: Solid Earth.}
  \DOI{10.1029/2018JB016451}
  \\
\Year{2018}  &
  \Me, \behn, \jao, \Yajing.
  Azimuthal seismic anisotropy of 70 Ma Pacific-plate upper mantle. 
  \emph{Journal of Geophysical Research: Solid Earth.}
  \DOI{10.1029/2018JB016451}
  \\
\end{EntriesTable}

%%%%%%%%%%%%%%%%%%%%%%%%%%%%%%%%%%%%%%%%%%%%%%%%%%%%%%%%%%%%%%%%%%%%%%%%%%%%%%%
\section{Reviewed editorials}

\begin{EntriesTable}
\Year{2023}  &
  \Me, \Thea, \GFun, \SHic, \CRowe, \STep, \JConv, \EKar, \RDCF.
  Editorial workflow of a community-led, all-volunteer scientific journal: lessons from the launch of Seismica.
  \emph{Seismica.}
  \DOI{10.26443/seismica.v2i2.1091} \\
\Year{2022}  &
  \CRowe, \MAgi, \JConv, \GFun, \CGal, \SHic, \THuy, \JLan, \TLec, \Me, \ROku, \Thea, \CRyc, \STep, \MvdE. 
  The launch of Seismica: a seismic shift in publishing. 
  \emph{Seismica.}
  \DOI{10.26443/seismica.v1i1.255}
\end{EntriesTable}



%%%%%%%%%%%%%%%%%%%%%%%%%%%%%%%%%%%%%%%%%%%%%%%%%%%%%%%%%%%%%%%%%%%%%%%%%%%%%%%
%\section{Manuscripts in progress}
%
%\begin{EntriesTable}
%\Year{in prep}  &
%  \Me, \danl, \Doug.
%  Constraints on bend-faulting and mantle hydration at the Marianas Trench from seismic anisotropy.
%  \small\emph{Submitted, copies available on request}
%\end{EntriesTable}



%%%%%%%%%%%%%%%%%%%%%%%%%%%%%%%%%%%%%%%%%%%%%%%%%%%%%%%%%%%%%%%%%%%%%%%%%%%%%%%
\section{Funding}

\begin{EntriesTable}
  \Duration{2023}{2024}  &
  NSF-OCE: ``An assessment of low-temperature, ductile lithospheric deformation using existing broadband seismic data from around South Island, New Zealand''. \newline
  Award ID: \href{https://www.nsf.gov/awardsearch/showAward?AWD_ID=2316757}{2316757}.
  PI: \Me. \newline
  \small \$84k to HF Mark
  \\
  \Duration{2023}{2025}  &
  NSF-OCE: ``Collaborative Research: Resolving the Origin of the Jurassic Quiet Zone''. \newline
  Award ID: \href{https://www.nsf.gov/awardsearch/showAward?AWD_ID=2221814}{2221814}.
  PI: \Masako, co-PIs: \Me; \WSage. \newline
  \small \$827k to WHOI including \$192k to HF Mark
  %\\
\end{EntriesTable}


%%%%%%%%%%%%%%%%%%%%%%%%%%%%%%%%%%%%%%%%%%%%%%%%%%%%%%%%%%%%%%%%%%%%%%%%%%%%%%%
\section{Awards}

\begin{EntriesTable}
  \Year{2019} & \textbf{Fossett Postdoctoral Fellowship}, \WUSTL
  \\
  \Year{2019} & \textbf{AGU Outstanding Student Paper Award (OSPA)}, Tectonophysics
  \\
  \Year{2016} & \textbf{AGU OSPA}, Tectonophysics
  \\
  \Year{2016} & \textbf{GeoPrisms OSPA}, honorable mention
  \\
  \Year{2015} & \textbf{National Science Foundation Graduate Research Fellowship}
  \\
  \Year{2014} & \textbf{Rosenblith Presidential Fellowship}, \MIT
  \\
  \Year{2014} & \textbf{Phi Beta Kappa}, Beta of Illinois
  \\
  \Year{2013} & \textbf{John Crerar Science Writing Prize}, \UChi
\end{EntriesTable}


%%%%%%%%%%%%%%%%%%%%%%%%%%%%%%%%%%%%%%%%%%%%%%%%%%%%%%%%%%%%%%%%%%%%%%%%%%%%%%%
\section{Teaching Experience}

\begin{EntriesTable}
  \Year{2023} &
  \textbf{WHOI Software Carpentries Python workshop}, Course assistant
  \\
  \Year{2020} &
  \textbf{EPS 564: Multidisciplinary Study of Subduction Zones}, Washington University in St. Louis, Guest lecturer
  \\
  \Year{2018} &
  \textbf{Kaufman Teaching Certificate}, MIT Teaching and Learning Lab
  \\
  \Year{2018} &
  \textbf{WHOI Software Carpentries git workshop}, Course assistant
  \\
  \Year{2017} &
  \textbf{MS-3221: Oceanography}, Massachusetts Maritime Academy, Co-Instructor
  \\
  \Year{2016} &
  \textbf{12.710: Elements of Modern Oceanography}, \JP, TA
  \\
  \Year{2016--2017} &
  \textbf{\mathrev}, Lecturer
\end{EntriesTable}


%%%%%%%%%%%%%%%%%%%%%%%%%%%%%%%%%%%%%%%%%%%%%%%%%%%%%%%%%%%%%%%%%%%%%%%%%%%%%%%
\section{Academic Service}

\begin{EntriesTable}
  \Duration{2022}{\Ongoing} & \href{https://seismica.library.mcgill.ca}{\textbf{\textit{Seismica}}} journal management committee, Co-chair of Standards and Copy Editing
  \\
  \Duration{2022}{\Ongoing} & \textbf{WHOI Postdoc Association}, Department representative
  \\
  \Duration{2022}{\Ongoing} & \textbf{WHOI Workplace Climate Committee}, Postdoc representative
  \\
  \Duration{2022}{\Ongoing} & \textbf{WHOI G\&G Proposal Club}, Co-organizer
  \\
  \Year{2021} & \textbf{WUSTL EPS Department URGE pod}, Pod leader
  \\
  \Duration{2018}{2019} & \textbf{500 Women Scientists of Cape Cod}, Steering Committee member
  \\
  \Duration{2015}{2019} & \textbf{\JP\ Farrington Collection}, Librarian
   \\
  \Duration{2014}{2016} & \textbf{\mathrev}, Coordinator
  \\
  ~ & \textbf{Peer reviewer for:} JGR Solid Earth, Nature Geoscience, Geophysical Research Letters, Science Advances, Nature Communications, Journal of Open Source Software, Geoscience Letters, Earth and Planetary Science Letters
  \\
  ~ & \textbf{Conference sessions convened:} SSA Annual Meeting 2022, \textit{Marine Seismology}; AGU Fall Meeting 2021, \textit{EP15 -- Coupling of the Cryosphere, Solid Earth, Surface, and Climate in Shaping Late Cenozoic Topography}
\end{EntriesTable}


%%%%%%%%%%%%%%%%%%%%%%%%%%%%%%%%%%%%%%%%%%%%%%%%%%%%%%%%%%%%%%%%%%%%%%%%%%%%%%%%
\section{Invited Seminars}

\begin{EntriesTable}
\Year{2023}  & \textbf{McGill University}, EPS Department seminar
  \\
  ~ & \textbf{University of Arizona}, Geosciences Colloquium
  \\
\Year{2022}  & \textbf{University of Washington School of Oceanography}
  \\
  ~ & \textbf{University of Wisconsin Madison}, Weeks Seminar
  \\
  ~ & \textbf{Syracuse University}, EES Department Colloquium
  \\
\Year{2021} & \textbf{University of Hawai'i Manoa}, Earth Sciences Department seminar
  \\
\Year{2019} & \textbf{University of Illinois Urbana-Champaign}, Brown bag seminar
  \\
\Year{2018} & \textbf{Lamont-Doherty Earth Observatory}, MGG/SGT seminar
  \\
  ~ & \textbf{WHOI}, Interdisciplinary Biology Seminar Series on Acoustics
\end{EntriesTable}

%%%%%%%%%%%%%%%%%%%%%%%%%%%%%%%%%%%%%%%%%%%%%%%%%%%%%%%%%%%%%%%%%%%%%%%%%%%%%%%%
\section{Field Experience}

\begin{EntriesTable}
  \Year{2020} & R/V \textit{Marcus G. Langseth}, MGL2003. \newline
  \small Active-source reflection and refraction survey in the Central Aleutians
  \\
  \Year{2017} & R/V \textit{Neil Armstrong}, AR23-02. \newline
  \small Collecting underway data along a ridge flowline in the Atlantic
  \\
  \Year{2016} & R/V \textit{Neil Armstrong}, AR05. \newline
  \small Scientific validation cruise
  \\
  \Year{2015} & ENAM land seismic experiment, North Carolina. \newline
  \small RT-125 deployment and recovery for active-source seismic lines on the Eastern North American Margin
  \\
  \Year{2014} & PRIDE SeisORZ seismic experiment, Botswana. \newline
  \small RT-125 deployment and recovery for an active-source seismic line across the Okavango Delta
  \\
\end{EntriesTable}


%%%%%%%%%%%%%%%%%%%%%%%%%%%%%%%%%%%%%%%%%%%%%%%%%%%%%%%%%%%%%%%%%%%%%%%%%%%%%%%%
\section{Selected Communication and Outreach}

\begin{EntriesTable}
  \Duration{2020}{2022} & Contributing writer and peer editor at \Website{geobites.org} \newline
  \small Blogging, general audience summaries of new research
  \\
  \Year{2018} & \href{https://www.whoi.edu/oceanus/feature/hannah-mark/}{How is the seafloor made?} \newline
  \small Magazine article, \textit{Oceanus} Vol 53. No. 2
  \\
  \Year{2017} & (Way) under the sea: Imaging the rocks beneath the deep ocean \newline
  \small Public talk, Science Made Public lecture series, Woods Hole, MA
  \\
  ~ & The lithosphere-asthenosphere boundary: What it is, where it is, and why you should care \newline
  \small Public talk, Woods Hole Public Library, Woods Hole, MA
  \\
  \Year{2016} & Under-under the sea: Imaging the rocks beneath the deep ocean \newline
  \small Public talk, Woods Hole Public Library, Woods Hole, MA
  \\
\end{EntriesTable}



%%%%%%%%%%%%%%%%%%%%%%%%%%%%%%%%%%%%%%%%%%%%%%%%%%%%%%%%%%%%%%%%%%%%%%%%%%%%%%%%
\section{Conference Abstracts \small (*=invited)}
\begin{EntriesTable}
\Year{2023} &
  Gourdeau, A, Prush, V, \CRowe, Wang, K, Rosset, P, Chouinard, L, Lamothe, M, \Me, Morris, I, Laly, M, Nackers, C.
  \textit{An Ongoing Search for Active Faults in the Western Quebec Seismic Zone, Eastern Canada.}
  \AGU, San Francisco, CA. \\
\Year{2022} &
  *\Me, \jac, \danl, \Jim, \Greg, \RobE, \behn.
  \textit{Where is the G discontinuity, what does it represent, and how can we tell? A case study from the NoMelt experiment.}
  \AGU, Chicago, IL.
  \\
  ~ &
  \Ari, \Donna, \danl, \Me, \JEst, Boston, B.
  \textit{Seismic reflection imaging of along-strike changes in forearc structure in the Andreanof segment of the Aleutian subduction zone.}
  \AGU, Chicago, IL.
  \\
  ~ &
  \Josh, \Jim, \Me, \Greg, \LHans, \danl, \jac, \RobE.
  \textit{Seismological Evidence for Girdled Olivine Fabric in Oceanic Lithosphere and Implications for Mantle Deformation Processes During Seafloor Spreading.}
  \AGU, Chicago, IL.
  \\
  ~ &
  \Doug, \Walid, \Me.
  \textit{Seismic Anisotropy and Mantle Dynamics Associated with Slab Windows.}
  \AGU, Chicago, IL.
  \\
  ~ &
  \Me, \Doug, \danl.
  \textit{Insights into bend-faulting and mantle hydration at the Marianas trench from seismic anisotropy.}
  SSA Annual Meeting, Belleview, WA (poster).
  \\
\Year{2021} &
  \Me, \Doug, \Walid, \Ivins, \ARich, \Bix, \EMard, \Rodrigo, \SBarr.
  \textit{Thermal erosion of the lithosphere in the Patagonian slab window and implications for glacial isostatic adjustment.}
  \AGU, New Orleans, LA.
  \\
  ~ &
  \Me, \danl, \Donna, \Ari, \JEst.
  \textit{Crustal structure along-strike in the Andreanof segment of the Aleutian Arc from wide-angle seismic refraction data.}
  \AGU, virtual (poster).
  \\
  ~ &
  \Doug, \Walid, \Me, \Patrickelder, \ARich, \SBarr.
  \textit{Geodynamics of the Patagonian Slab Window Constrained by Shear Wave Splitting and Seismic Imaging.}
  \AGU, New Orleans, LA.
  \\
  ~ &
  \Me, \Doug, \danl.
  \textit{Estimating bend-faulting and mantle hydration at the Marianas trench from seismic anisotropy.}
  EGU General Assembly, virtual.
  \\
  ~ &
  *\Me.
  \textit{Anisotropy in slightly more than one dimension.}
  Marine Seismology Symposium, virtual.
  \\
  ~ &
  \Me, \Doug, \danl.
  \textit{Estimating bend-faulting and mantle hydration at the Marianas trench from seismic anisotropy.}
  Marine Seismology Symposium, virtual (poster).
  \\
\Year{2020} & 
  \Me, \Doug, \Maeva, \Bix, \Ivins, \ARich, \SBarr.
  \textit{Seismic structure and the extent of the slab window beneath the Northern and Southern Patagonian Icefields.}
  \AGU, virtual (poster).
  \\
\Year{2019} &
  \Me, \jac, \danl, \Greg, \Jim, \RobE.
  \textit{The depth, sharpness, and strength of the G discontinuity from S-to-p receiver functions at the NoMelt site in the central Pacific.}
  \AGU, San Francisco, CA.
  \\
\Year{2018} &
  \Me, \jac, \danl, \Greg, \Jim, \RobE.
  \textit{S-to-p receiver functions for the central Pacific from NoMelt.}
  \AGU, Washington, D. C.
  \\
  ~ &
  \Billy, \Me, \Fiona, \Emman, \jao, \Jianhua, \SBrow, \PSmal, \Yang, \Vero, \behn.
  \textit{Temporal variability of seafloor spreading processes documented along an 80-Myr geophysical transect across the Mid-Atlantic Ridge.}
  \AGU, Washington, D.C.
  \\
  ~ &
  \Jim, \Josh, \Me, \PLin, \ESar, \ZMa, \danl, \jac, \Greg, \RobE, \CDal.
  \textit{A comprehensive portrait of the central Pacific lithosphere-asthenosphere system from NoMelt seafloor geophysical observations.}
  \AGU, Washington, D.C.
  \\
\Year{2017} &
  *\Me, \behn, \Yajing, \jao.
  \textit{Geometric and thermal controls on normal fault seismicity.}
  \AGU, New Orleans, LA.
  \\
  ~ &
  \Me, \danl, \jac, \Nate, \Greg, \Jim, \RobE.
  \textit{Seismic anisotropy of 70 Ma Pacific-plate upper mantle.}
  \AGU, New Orleans, LA.
  \\
  ~ &
  *\Me, \behn, \Yajing, \jao.
  \textit{Seismic coupling at divergent plate boundaries from rate-and-state friction.}
  GeoPrisms TEI on Rift Initiation and Evolution, Albuquerque, NM.
  \\
\Year{2016} &
  \Me, \behn, \Yajing, \jao.
  \textit{Seismic coupling at divergent plate boundaries from rate-and-state friction.}
  \AGU, San Francisco, CA.
  \\
  ~ &
  \Me, \behn, \Yajing, \jao.
  \textit{Rate-and-state friction models of seismic cycles on oceanic normal faults.}
  GRC on Rock Deformation, Andover, NH (poster).
  \\
  ~ &
  \Me, \danl, \Jim, \jac, \Greg, \RobE, \PLin.
  \textit{Seismic anisotropy in the Pacific upper mantle.}
  Seismology Student Workshop, Lamont Doherty Earth Observatory.
  \\
\Year{2014} &
  \Me, \danl, \Jim, \jac, \Greg, \RobE.
  \textit{Pacific upper mantle seismic anisotropy from the active-source component of the NoMelt experiment.}
  \AGU, San Francisco, CA (poster).
  \\
\Year{2012} &
  \Dion, \Me.
  \textit{The Effect of Wavelength-Dependent Emissivity on the Melting Temperature of Iron from Shock Wave Measurements.}
  \AGU, San Francisco, CA (poster).
\end{EntriesTable}



%%%%%%%%%%%%%%%%%%%%%%%%%%%%%%%%%%%%%%%%%%%%%%%%%%%%%%%%%%%%%%%%%%%%%%%%%%%%%%%
%\section{Open Science}
%
%\begin{EntriesTable}
%  \Duration{2010}{\Ongoing} &
%  \textbf{Fatiando a Terra} | \Website{www.fatiando.org}
%  \newline
%  Python tools for geophysical data processing, forward modeling, and inversion
%  \\
%  \Duration{2017}{\Ongoing} &
%  \textbf{PyGMT} | \Website{www.pygmt.org}
%  \newline
%  A Python interface for the Generic Mapping Tools
%  \\
%  \Duration{2017}{\Ongoing} &
%  \textbf{The Generic Mapping Tools (GMT)} | \Website{www.generic-mapping-tools.org}
%  \newline
%  A data processing and mapping toolbox for the Earth, Ocean, and Planetary Science
%  \\
%  \Duration{2009}{2016} &
%  \textbf{Tesseroids} | \Website{tesseroids.leouieda.com}
%  \newline
%  Forward modeling of gravitational fields in spherical coordinates
%\end{EntriesTable}


\end{document}


